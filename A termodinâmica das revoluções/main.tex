\documentclass[11pt]{article}

\usepackage{sectsty}
\usepackage{graphicx}

% Margins
\topmargin=-0.40in
\evensidemargin=0in
\oddsidemargin=0in
\textwidth=6.5in
\textheight=9.0in
\headsep=0.25in

\begin{document}
\pagebreak



% Optional TOC
% \tableofcontents
% \pagebreak

%--Paper--

\section{A Termodinâmica das Revoluções}

\subsection{Parte I - Entropia, Lei Natural e Dharma.}

.

Quando fazemos algo com algum propósito, queremos atingir algum estado específico. O chamado "Aumento da entropia" é justamente o oposto disso. A entropia, matematicamente falando é tudo que não conseguimos discernir ou agrupar através de uma definição limitada, simples.\\\\

Quando um sistema aumenta a sua entropia, na verdade ele está aumentando a quantidade de informação desconhecida, "descontrolada", informação fora do espaço desejado.\\\\

Para deixar um sistema em um estado fixo, determinado, a gente gasta um excesso de energia. Esse excesso é a diferença em relação ao "estado natural" ou o "Dharma", por assim dizer, o caminho natural da realidade, que é o estado onde a entropia se torna naturalmente maior e mais incontrolável.\\\\

É onde o gelo derrete, é onde o gás se espalha, é onde as cartas se misturam. E também, é onde a árvore cresce, onde o rio desce, e onde a chuva cai. No entanto, se quisermos subir o rio, evaporar a chuva e organizar o baralho, sempre haverá um esforço adicional.\\\\

O esforço para algo se tornar diferente daquilo que estava determinado a ser sempre existirá, pois a natureza, em seu estado mais natural e livre de intervenções, o estado "Dharmático", é ótimo por definição. E qualquer coisa ótima, não pode economizar mais, pois já economizou tudo que podia.\\\\

Inclusive: A natureza converge em estados de economia por simples equilíbrio, é algo belo e elegante, mas a realidade não a busca por que é meramente bonito. Os cientistas e matemáticos enxergam como belo justamente por que a cognição humana está circunscrita nas mesmas leis de equilíbrio entre todas as partes.\\\\

\subsection{Parte II - O Karma do excesso.}
.

Voltando a falar do excesso: Ele pode ser perfeitamente ilustrado pelo mecanismo do Ar condicionado. É conhecimento popular que um Ar condicionado não pode refrigerar nada dentro de um ambiente totalmente fechado. Ele precisa trocar o ar com o lado de fora. \\\\

Isso é puro conhecimento analógico: O Ar-Condicionado, ao gelar o ar, precisa pelo menos gerar um calor igual do outro lado. Mas os físicos descobriram: sempre terá um calor extra adicional. Essa é a "segunda lei da termodinâmica". É o custo extra para reorganizar as moléculas em um estado mais calmo.\\\\

As consequências do controle e do caminho para se chegar a um estado parecem um beco sem saída, um cobertor curto. Se resolve um problema se criando outro, pior ainda. É o Karma em sua natureza, a ação que, buscando reduzir entropia, na verdade gera mais entropia. E que se afasta drasticamente do Dharma, do caminho natural, em dimensões cada vez mais catastróficas.\\\\ 


\subsection{Parte III - Os prelúdios da revolução.}
.

Porém, existem duas possibilidades que ainda não falamos: (1) Um sistema pode minimizar este excedente (2) Ou ainda o efeito reverso pode acontecer, quase como magicamente: alguma lei oculta é revelada onde um novo caminho natural é "desbloqueado" e novas formas de poder são emanadas de dentro do próprio sistema (!!!)\\\\

Vamos voltar ao primeiro caso. Explicando melhor com exemplos. Para a primeira possibilidade, observamos como um barco a vela funciona. Ele não tem um motor que precisa gastar um grande excesso de combustível, ele usa uma energia latente do vento, que é apenas sutilmente redirecionada. Claro, ainda o sistema gasta o caminho natural contra si mesmo, pois a vela gera atrito, o barco bate nas ondas, alguma energia está sendo desviada e não está sendo aproveitada para a finalidade, que é navegar para o destino.\\\\

Porém, o barco a vela usa o próprio caminho natural do vento de forma sutil, e assim consegue quase como mágica controlar a sua própria direção. E, em última instância, toda a forma de vida funciona desta forma. A vida, esta vida orgânica, requer sim muito dispêndio de energia, comida e destruição, o controle que gera mais entropia. Porém quanto mais sutis e complexos os organismos se tornam, mais sincrônicos e otimizados eles também se tornam. Isso se aplica, obviamente dentro de um organismo onde todas as células se comunicam, mas também dentro de ecossistemas cooperativos.\\\\

O Dharma de um não é usado contra o Dharma de outro. Os caminhos latentes naturais despertam um caminho em comum, que não seria possível se apenas os dois fizessem por si só. Esse é o conceito de sinergia, que obviamente não recua a entropia, mas certamente possibilita menos entropia por espaço, por volume, por seres. \textbf{A entropia total não precisa ser menor, mas a entropia por cada entidade se torna menor.}\\\\

Essa afirmação é muito importante e vou elaborar ela melhor depois.\\\\

\subsection{Parte IV - A possibilidade da revolução.}

.

Como falamos sobre \textbf{sinergia}, é agora que a segunda possibilidade pode ser anunciada de forma mais clara. Relembrando, é a possiblidade de um efeito reverso: a revelação de uma lei natural até então desconhecida que produz energia quase de forma mágica. E nada me vem à mente senão a energia atômica, usada nas usinas nucleares, e infelizmente, na bomba atômica. Explico:\\\\

Os átomos são condensados de informação, muito duros para revelar qualquer transformação interna. Nada - aparentemente - acontece dentro deles. Poderíamos afirmar que não tem sequer entropia sendo gerada, pois só acontece algo onde algo muda. \\\\

Essa casca ilusória isolada do resto do mundo não pode ser transformada, sequer tem alguma lei natural que a descreva internamente, ou melhor - há de forma muito sutil - se considerarmos um átomo de Urânio, que foi descoberto por nós como meio útil para extrair a energia atômica.\\\\

O Átomo de Urânio decai, emitindo pequenas partes de si mesmo, é aleatório, é auto-determinado. Aí está uma pista do fluxo natural da realidade, indomável e até então incompreensível. Como poderíamos nós, extrair algo útil dessa natureza tão sutil e incotrolável?\\\\

Pois então, assim como conseguimos içar as velas e usar a linguagem para cooperar com as pessoas, os Átomos podem conversar por meio de elaboradas armadilhas que colocamos para que eles se incitem em uma reação em cadeia.\\\\

\subsection{Parte V - O sistema natural contra si mesmo}

.

Aquilo que era pequeno, se torna maior, que se torna maior ainda. Esse processo - se pensarmos profundamente - é extremamente espantoso, assustador, e muito belo. E revela uma natureza fundamental da realidade - todas as leis aparentes que chamamos de realidade ou "lei natural" podem ser transformadas, ou até mesmo "revertidas" através do preciso entendimento das leis e do uso delas "contra" elas mesmas.\\\\

É o que os Yogis chamam de usar a própria mente para transcender a própria mente. Na verdade é, de forma precisa e muito coerente, quase como numa alquimia geométrica, compreender as sinergias necessárias entre as partes do todo para revelar partes menores, para desmontar relações ocultas e desconhecidas até então. É a própria retirada do véu de Maia que revela o que estava imbricado e compactado por meio de repetidos acúmulos "Karmáticos", marcas aparentemente irreversíveis. \\\\


\subsection{Parte VI - A magnitude}
.

E aí que voltamos ao ponto que quero fechar aqui: Em todos esses processos, inclusive no da energia atômica, ainda é necessário energia. Ainda será gerada entropia. Mas o que mais importa, e isso é o que sempre importa no nosso mundo afinal de contas: \textbf{Quanto?}\\\


É possível um bater de asas de uma borboleta causar um furacão? A impossibilidade não mora no fenômeno, transformações são transformações, independente da escala. "Me dê uma alavanca e eu conseguirei levantar o mundo" disse Arquimedes.\\\\\

A escala é uma mera ilusão abstraída pela nossa mente, pela nossa dificuldade de compreender as dimensões do universo. Mas toda transformação é essencialmente \textbf{A transformação} (em abstrato), uma mudança que pode e irá nos chocar por que pode ter sido colocada de forma tão bem (ou mal!) intencionada que faz um átomo gerar um sol em um instante. \\\\

Porém, para nossa vida ordinária é tudo. Dois peixes podem se tornar alimento para milhares de pessoas. Estava Jesus fazendo milagre ou apenas ensinando algo mais profundo: Em pouco tempo, apenas dois peixes podem se tornar centenas de peixes, se forem cultivados adequadamente. A vida orgânica nos ensina isso a todo momento!\\\\

\subsection{Parte VII - A verdadeira magia}

.

Então, aqui resgato a afirmação que guardávamos: \textbf{A entropia total não precisa ser menor, o que importa é a entropia por cada unidade de ser, espaço, tempo, ser menor.}\\\\

Em outras palavras, o ruído, a sujeira e o caos serão inevitáveis. Mas a chave está em minimizar a loucura, a incoerência e o sofrimento \textbf{especificamente para cada pequeno ser do universo, seja lá quão pequeno ele for.}\\\\

E assim torna-se claro que diversos princípios enumerados já em tradições antigas da sociedade humana não é uma mera invenção arbitrária ou um desejo idealista. É uma necessidade energética, real e urgente.\\\\

A vida e a história humana nos ensinou. A vida só é bem sucedida graças à sua alquimia geométrica, capaz de promover, de forma cada vez mais crescente e coerente, a sinergia entre células, órgãos e espécies.\\\\

A história humana mostrou, por sua vez, a capacidade de transcender as leis com as próprias leis e com isso descobrir desbloquear potências em escalas inimagináveis. A revolução agrícola, a revolução industrial, a descoberta do petróleo, e a era atômica são apenas exemplos valorizados pela nossa sociedade focada em produção e consumo.\\


Mas ao olharmos em todos os territórios da humanidade também conseguimos observar essa mesma revolução em diversos níveis: a revolução da linguagem, a revolução da escrita, a revolução das artes, que no fim desembocam na revolução mais profunda: a revolução da consciência. É aqui onde famosamente os Yogis e Budistas se encontram, aqueles que ousaram usar a ferramenta da mente contra si mesma para desbloquear potenciais latentes e até então inconscientes.\\\\

Poderia passar horas enumerando os exemplos, mas fica claro uma coisa: sempre haverá possibilidades de "quebrar" as leis da termodinâmica de formas inimagináveis. E também há uma necessidade real: se quisermos sobreviver, ou pelo menos viver com menos sofrimento, precisamos nos organizar e achar nossas sinergias ocultas, de forma simultânea e integrada.
%--/Paper--

\end{document}
