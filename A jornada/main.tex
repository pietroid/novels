\documentclass[11pt]{article}

\usepackage{sectsty}
\usepackage{graphicx}

% Margins
\topmargin=-0.40in
\evensidemargin=0in
\oddsidemargin=0in
\textwidth=6.5in
\textheight=9.0in
\headsep=0.25in

\title{ A Jornada }
\author{ Pietro Domingues }
\date{\today}

\begin{document}
\maketitle	
\pagebreak



% Optional TOC
% \tableofcontents
% \pagebreak

%--Paper--

\section{O trem}

\textbf{O Yogue:} A vida é como andar de trem. A gente observa tudo que está passando ao nosso lado. Conseguimos olhar um pouco para trás e um pouco para frente, se preparar para a chegada de uma estação ou repousar após embarcar de outra. \\

Porém, sempre estaremos viajando, naquele exato momento, naquele exato lugar. Podemos optar por pensar em todas as estações que virão à frente e imaginar o que teria acontecido se desembarcássemos nas que já passaram. Mas podemos simplesmente olhar para a janela e aproveitar a paisagem.\\

\textbf{O Cientista:} Poeticamente, está metáfora é muito linda, mas é ingênuo pensar que no mundo contemporâneo não é possível ter sempre essa postura. Precisamos agir, reagir, prever, ter compromissos externos. Parece mais que estamos num carrinho de bate-bate e precisamos desviar ou então impor nossa força, a todo custo.\\

\textbf{O Yogue:} Ora, mas isso me lembra o mecanismo molecular dos gases. Quanto mais energia a gente bota neles, mais eles batem e se rebatem. É assim o nosso mundo, e isso é verdade. E é assim que está nossa mente, também. Estamos vivendo em uma panela de pressão, sempre prestes a explodir. Cada vez mais energia está sendo condensada em um mesmo espaço confinado.\\

\textbf{O Cientista:} Exatamente! E conforme a lei dos gases, não temos como escapar disso, pois só haverá mais pressão! Para sobrevivermos, devemos sempre calcular para onde vamos.\\

\textbf{O Yogue:} E como o avião voa? \\

\textbf{O Cientista:} Por meio do ar?\\

\textbf{O Yogue:} Pois é! O avião é grande o suficiente para conseguir direcionar, mesmo que seja por meio da fricção e da colisão, as moléculas de ar, e nessa mágica toda, ele cria uma pressão direcionada que o faz se sustentar. O avião não tem medo do choque, ao invés de calcular todas as partículas, o que seria impossível, ele simplesmente tem potência o suficiente e resistência para conseguir enfrentar o que é necessário e converter em energia útil para seu objetivo.\\

\section{A viagem}

Então o Yogue, entrando em um profundo estado de respiração, diz para o jovem cientista:\\

Agora quero te mostrar uma coisa. Ambos tomam um líquido espesso e tocam a superfície de um metal muito polido. \\

Naquele momento, tudo para. Eles entram numa espécie de buraco negro, luminosamente vazio. Dentro dele, a consciência deles é reduzida a uma coisa singular.\\

\textbf{O Yogue:} Agora somos apenas esta pequena esfera viajando pelo universo. Quero te mostrar a nossa origem.\\

E então, em um instante, o cientista se encontra completamente sozinho em uma floresta recheada dos mais diversos estímulos. Os cheiros são variados, os sons, mais ainda. Porém tudo parece viver em estranha harmonia.\\

O que mais choca ele no entanto é ao tocar no seu próprio rosto. Ele próprio tinha barbas, um rosto enrugado e fino. Suas roupas eram poucas, quase nenhuma. No entanto estava se sentindo íntegro, lúcido e dentro de si mesmo.\\

Ele viveu durante anos sem mesmo suspeitar sobre o que o Yogue tinha falado. Ele morreu meditando, em um estado de suprema harmonia com cada vibração em sua volta.\\

Acordando assustado, ele então se vê em um transe, no meio de tambores e cantos. A floresta era a mesma, porém os ritmos e as pessoas eram diferentes. Tudo continuava novo, tudo continuava vivo.\\

Ele não havia morrido. Só estava num transe dentro da própria cura que estava sendo realizada por estes Shamans.\\ 

Depois de um instante, após ouvir um longo e distante canto de pássaro se tornar mais e mais forte, então ele se transforma em um e sai voando por planícies e planícies afora, até que repousa em um vulcão.\\

\end{document}